\section{Introduction}
We describe 
a novel approach to supervised feature selection 
based on 
neural estimation of mutual information 
between
features 
and 
targets.
We call our method 
\emph{Minerva}.


Feature selection methods
are distinguished in
two classes: 
wrappers,
and
filters.
On the one hand,
wrappers
assume 
knowledge of the learning model
and use the learning process as a subroutine. 
They are usually computationally expensive,
and they are model-dependent.
On the other hand,
filters 
utilize a score of dependence between features and target,
and they select a subset of features based on this score.
Filters
do not require knowledge of the learning procedure,
and hence they are model-independent.

Minerva belongs to the class of filters,
and utilizes the mutual information as score.

Estimating the mutual information between random variables
is challenging. 
The classical estimator is the
Kraskov-St{\"o}gbauer-Grassberger (KSG) estimator
introduced in \cite{KSG04est},
and proved to be a consistent estimator in
\cite{GOV18dem}.
Recently, 
a modern, consistent estimator 
called 
\emph{Mutual Information Neural Estimator (MINE)}
was proposed
in
\cite{BBROBCH18mut},
and applications have flourished.
Our feature selection procedure Minerva
utilises MINE to compute the mutual information score. 
Our score
evaluates subsets of features as an ensemble,
instead of considering one feature at a time
as most feature selection filters do.
This allows us to 
capture sophisticated relationships
between features and targets,
and 
to take such sophisticated relationships
into account when selecting relevant features.
We give examples of such relationships,
and we demonstrate that
Minerva is capable of performing an exact selection,
whereas other existing methods fail to do so.


The paper is organised as follows.
Section \ref{sec.mine}
recalls the fundmentals of MINE,
the Mutual Information Neural Estimator. 
Section \ref{sec.method}
explains our method of featurte selection 
and
describe our neural network architecture.
Section \ref{sec.experiments}
presents our numerical experiments with Minerva.
Finally, 
Section \ref{sec.proofs}
collects the proofs of the lemmata and propositions of the article.



