\section{Neural estimation of mutual information}
\label{sec.mine}

In this section,
we recall the neural estimation of mutual information
following
\cite{BBROBCH18mut}.




Let 
$\spaceX \subset \Rd$
and 
$\spaceY \subset \R^{e}$
represent sample spaces.
Let 
$X$
and 
$Y$
be random variables taking values in 
$\spaceX$
and 
$\spaceY$
respectively.
Let 
$\Prob_{XY}$ 
denote the joint distribution of 
$X$
and
$Y$; 
and let 
$\Prob_{X} \otimes \Prob_{Y}$
denote the
product of the marginal
laws of $X$ and $Y$.

The mutual information 
$I(X; Y)$
of 
$X$ 
and 
$Y$
is defined as the 
Kullback-Leibler
divergence 
between
$\Prob_{XY}$ 
and
$\Prob_{X} \otimes \Prob_{Y}$.

Using the Donsker-Varadhan representation of the 
Kullback-Leibler
divergence 
(see \cite{DV83asy}),
we can write
\begin{equation}
	\label{eq.dvrepresentationofmutualinfo}
	I(X; Y)
	= 
	\sup_{f}
	\quad
	\Expectation_{\Prob_{XY}}
	\left[
		f(X, Y)
	\right]
        - \log
	\left(
	\Expectation_{\Prob_{X} \otimes \Prob_{Y}}
	\left[
		\exp(f(X, Y))
	\right]
	\right),
\end{equation}
where 
$\Expectation_{\Prob_{XY}}$ 
denotes expectation with respect to 
$\Prob_{XY}$,
$\Expectation_{\Prob_{X} \otimes \Prob_{Y}}$ 
denotes expectation with respect to 
$\Prob_{X} \otimes \Prob_{Y}$,
and
the supremum is taken over 
all measurable functions 
$f: \spaceX \times \spaceY \rightarrow \R$ 
such that the two expectations are finite.

Given samples 
\begin{equation}
	\label{eq.jointsamples}
	(x_1, y_1), 
	\dots,
	(x_n, y_n), 
\end{equation}
from the joint distribution of $X$ and $Y$,
we can use the representation
in equation \eqref{eq.dvrepresentationofmutualinfo}
to 
estimate the mutual information of the two random variables.
Indeed,
we can  represent the functions
$f$
in equation \eqref{eq.dvrepresentationofmutualinfo}
via a neural network 
$f_\theta$ 
parametrised by $\theta \in \Theta$,
and then 
run gradient ascend in the parameter space $\Theta$ to
maximise 
the emprirical objective functional of equation \eqref{eq.dvrepresentationofmutualinfo}, 
where
the first expectation is replaced by 
\begin{equation*}
	\frac{1}{n}\sum_{i=1}^{n} f(x_i, y_i),
\end{equation*}
and the second expectation is replaced by 
\begin{equation*}
	\frac{1}{n}\sum_{i=1}^{n} \exp\left(
	f(x_i, y_{\sigma(i)})
	\right),
\end{equation*}
where $\sigma$ is a permutation 
used to shuffle 
the $Y$-samples 
and 
hence turn the samples of equation \eqref{eq.jointsamples}
into samples from 
$\Prob_{X} \otimes \Prob_{Y}$.

The described approach to 
estimate the mutual information 
$I(X; Y)$
is at the core of MINE, 
and we will rely on this method 
to construct a feature selection filter. 








